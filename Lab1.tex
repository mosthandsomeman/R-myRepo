% Options for packages loaded elsewhere
\PassOptionsToPackage{unicode}{hyperref}
\PassOptionsToPackage{hyphens}{url}
%
\documentclass[
]{article}
\usepackage{lmodern}
\usepackage{amssymb,amsmath}
\usepackage{ifxetex,ifluatex}
\ifnum 0\ifxetex 1\fi\ifluatex 1\fi=0 % if pdftex
  \usepackage[T1]{fontenc}
  \usepackage[utf8]{inputenc}
  \usepackage{textcomp} % provide euro and other symbols
\else % if luatex or xetex
  \usepackage{unicode-math}
  \defaultfontfeatures{Scale=MatchLowercase}
  \defaultfontfeatures[\rmfamily]{Ligatures=TeX,Scale=1}
\fi
% Use upquote if available, for straight quotes in verbatim environments
\IfFileExists{upquote.sty}{\usepackage{upquote}}{}
\IfFileExists{microtype.sty}{% use microtype if available
  \usepackage[]{microtype}
  \UseMicrotypeSet[protrusion]{basicmath} % disable protrusion for tt fonts
}{}
\makeatletter
\@ifundefined{KOMAClassName}{% if non-KOMA class
  \IfFileExists{parskip.sty}{%
    \usepackage{parskip}
  }{% else
    \setlength{\parindent}{0pt}
    \setlength{\parskip}{6pt plus 2pt minus 1pt}}
}{% if KOMA class
  \KOMAoptions{parskip=half}}
\makeatother
\usepackage{xcolor}
\IfFileExists{xurl.sty}{\usepackage{xurl}}{} % add URL line breaks if available
\IfFileExists{bookmark.sty}{\usepackage{bookmark}}{\usepackage{hyperref}}
\hypersetup{
  pdftitle={Lab 1: Data Manipulation, Random Number Generation},
  hidelinks,
  pdfcreator={LaTeX via pandoc}}
\urlstyle{same} % disable monospaced font for URLs
\usepackage[margin=1in]{geometry}
\usepackage{color}
\usepackage{fancyvrb}
\newcommand{\VerbBar}{|}
\newcommand{\VERB}{\Verb[commandchars=\\\{\}]}
\DefineVerbatimEnvironment{Highlighting}{Verbatim}{commandchars=\\\{\}}
% Add ',fontsize=\small' for more characters per line
\usepackage{framed}
\definecolor{shadecolor}{RGB}{248,248,248}
\newenvironment{Shaded}{\begin{snugshade}}{\end{snugshade}}
\newcommand{\AlertTok}[1]{\textcolor[rgb]{0.94,0.16,0.16}{#1}}
\newcommand{\AnnotationTok}[1]{\textcolor[rgb]{0.56,0.35,0.01}{\textbf{\textit{#1}}}}
\newcommand{\AttributeTok}[1]{\textcolor[rgb]{0.77,0.63,0.00}{#1}}
\newcommand{\BaseNTok}[1]{\textcolor[rgb]{0.00,0.00,0.81}{#1}}
\newcommand{\BuiltInTok}[1]{#1}
\newcommand{\CharTok}[1]{\textcolor[rgb]{0.31,0.60,0.02}{#1}}
\newcommand{\CommentTok}[1]{\textcolor[rgb]{0.56,0.35,0.01}{\textit{#1}}}
\newcommand{\CommentVarTok}[1]{\textcolor[rgb]{0.56,0.35,0.01}{\textbf{\textit{#1}}}}
\newcommand{\ConstantTok}[1]{\textcolor[rgb]{0.00,0.00,0.00}{#1}}
\newcommand{\ControlFlowTok}[1]{\textcolor[rgb]{0.13,0.29,0.53}{\textbf{#1}}}
\newcommand{\DataTypeTok}[1]{\textcolor[rgb]{0.13,0.29,0.53}{#1}}
\newcommand{\DecValTok}[1]{\textcolor[rgb]{0.00,0.00,0.81}{#1}}
\newcommand{\DocumentationTok}[1]{\textcolor[rgb]{0.56,0.35,0.01}{\textbf{\textit{#1}}}}
\newcommand{\ErrorTok}[1]{\textcolor[rgb]{0.64,0.00,0.00}{\textbf{#1}}}
\newcommand{\ExtensionTok}[1]{#1}
\newcommand{\FloatTok}[1]{\textcolor[rgb]{0.00,0.00,0.81}{#1}}
\newcommand{\FunctionTok}[1]{\textcolor[rgb]{0.00,0.00,0.00}{#1}}
\newcommand{\ImportTok}[1]{#1}
\newcommand{\InformationTok}[1]{\textcolor[rgb]{0.56,0.35,0.01}{\textbf{\textit{#1}}}}
\newcommand{\KeywordTok}[1]{\textcolor[rgb]{0.13,0.29,0.53}{\textbf{#1}}}
\newcommand{\NormalTok}[1]{#1}
\newcommand{\OperatorTok}[1]{\textcolor[rgb]{0.81,0.36,0.00}{\textbf{#1}}}
\newcommand{\OtherTok}[1]{\textcolor[rgb]{0.56,0.35,0.01}{#1}}
\newcommand{\PreprocessorTok}[1]{\textcolor[rgb]{0.56,0.35,0.01}{\textit{#1}}}
\newcommand{\RegionMarkerTok}[1]{#1}
\newcommand{\SpecialCharTok}[1]{\textcolor[rgb]{0.00,0.00,0.00}{#1}}
\newcommand{\SpecialStringTok}[1]{\textcolor[rgb]{0.31,0.60,0.02}{#1}}
\newcommand{\StringTok}[1]{\textcolor[rgb]{0.31,0.60,0.02}{#1}}
\newcommand{\VariableTok}[1]{\textcolor[rgb]{0.00,0.00,0.00}{#1}}
\newcommand{\VerbatimStringTok}[1]{\textcolor[rgb]{0.31,0.60,0.02}{#1}}
\newcommand{\WarningTok}[1]{\textcolor[rgb]{0.56,0.35,0.01}{\textbf{\textit{#1}}}}
\usepackage{graphicx,grffile}
\makeatletter
\def\maxwidth{\ifdim\Gin@nat@width>\linewidth\linewidth\else\Gin@nat@width\fi}
\def\maxheight{\ifdim\Gin@nat@height>\textheight\textheight\else\Gin@nat@height\fi}
\makeatother
% Scale images if necessary, so that they will not overflow the page
% margins by default, and it is still possible to overwrite the defaults
% using explicit options in \includegraphics[width, height, ...]{}
\setkeys{Gin}{width=\maxwidth,height=\maxheight,keepaspectratio}
% Set default figure placement to htbp
\makeatletter
\def\fps@figure{htbp}
\makeatother
\setlength{\emergencystretch}{3em} % prevent overfull lines
\providecommand{\tightlist}{%
  \setlength{\itemsep}{0pt}\setlength{\parskip}{0pt}}
\setcounter{secnumdepth}{-\maxdimen} % remove section numbering

\title{Lab 1: Data Manipulation, Random Number Generation}
\author{}
\date{\vspace{-2.5em}September 21, 2020}

\begin{document}
\maketitle

Today's agenda: Manipulating data objects; using the built-in functions,
doing numerical calculations, and basic plots; reinforcing core
probabilistic ideas.

\begin{enumerate}
\def\labelenumi{\arabic{enumi}.}
\setcounter{enumi}{-1}
\tightlist
\item
  Open a new R Markdown file; set the output to HTML mode and ``Knit''.
  This should produce a web page with the knitting procedure executing
  your code blocks. You can edit this new file to produce your homework
  submission.
\end{enumerate}

\hypertarget{background}{%
\subsection{Background}\label{background}}

The exponential distribution is defined by its cumulative distribution
function

\[F(x) = 1-e^{-\lambda x}\]

The R function \texttt{rexp} generates random variables with an
exponential distribution.

\begin{Shaded}
\begin{Highlighting}[]
\KeywordTok{rexp}\NormalTok{(}\DataTypeTok{n=}\DecValTok{10}\NormalTok{, }\DataTypeTok{rate=}\DecValTok{5}\NormalTok{)}
\end{Highlighting}
\end{Shaded}

\begin{verbatim}
##  [1] 0.175010043 0.003414805 0.095499582 0.046811053 0.024805705 0.202061963
##  [7] 0.227913643 0.127331554 0.319665699 0.027383198
\end{verbatim}

produces 10 exponentially-distributed numbers with rate (\(\lambda\)) of
5. If the second argument is omitted, the default rate is 1; this is the
\texttt{standard\ exponential\ distribution}.

\hypertarget{part-i}{%
\subsection{Part I}\label{part-i}}

\begin{enumerate}
\def\labelenumi{\arabic{enumi}.}
\item
  Generate 200 random values from the standard exponential distribution
  and store them in a vector \texttt{exp.draws.1}. Find the mean and
  standard deviation of \texttt{exp.draws.1}.
\item
  Repeat, but change the rate to 0.1, 0.5, 5 and 10, storing the results
  in vectors called \texttt{exp.draws.0.1}, \texttt{exp.draws.0.5},
  \texttt{exp.draws.5} and \texttt{exp.draws.10}.
\item
  The function \texttt{plot()} is the generic function in R for the
  visual display of data. \texttt{hist()} is a function that takes in
  and bins data as a side effect. To use this function, we must first
  specify what we'd like to plot.

  \begin{enumerate}
  \def\labelenumii{\alph{enumii}.}
  \tightlist
  \item
    Use the \texttt{hist()} function to produce a histogram of your
    standard exponential distribution.
  \item
    Use \texttt{plot()} with this vector to display the random values
    from your standard distribution in order.
  \item
    Now, use \texttt{plot()} with two arguments -- any two of your other
    stored random value vectors -- to create a scatterplot of the two
    vectors against each other.
  \end{enumerate}
\item
  We'd now like to compare the properties of each of our vectors. Begin
  by creating a vector of the means of each of our five distributions in
  the order we created them and saving this to a variable name of your
  choice. Using this and other similar vectors, create the following
  scatterplots:

  \begin{enumerate}
  \def\labelenumii{\alph{enumii}.}
  \tightlist
  \item
    The five means versus the five rates used to generate the
    distribution.
  \item
    The standard deviations versus the rates.
  \item
    The means versus the standard deviations.
  \end{enumerate}
\end{enumerate}

For each plot, explain in words what's going on.

\hypertarget{part-ii}{%
\subsection{Part II}\label{part-ii}}

\begin{enumerate}
\def\labelenumi{\arabic{enumi}.}
\setcounter{enumi}{4}
\tightlist
\item
  R's capacity for data and computation is large to what was available
  10 years ago.

  \begin{enumerate}
  \def\labelenumii{\alph{enumii}.}
  \tightlist
  \item
    To show this, generate 1.1 million numbers from the standard
    exponential distribution and store them in a vector called
    \texttt{big.exp.draws.1}. Calculate the mean and standard deviation.
  \item
    Plot a histogram of \texttt{big.exp.draws.1}. Does it match the
    function \(1-e^{-x}\)? Should it?
  \item
    Find the mean of all of the entries in \texttt{big.exp.draws.1}
    which are strictly greater than 1. You may need to first create a
    new vector to identify which elements satisfy this.
  \item
    Create a matrix, \texttt{big.exp.draws.1.mat}, containing the the
    values in \texttt{big.exp.draws.1}, with 1100 rows and 1000 columns.
    Use this matrix as the input to the \texttt{hist()} function and
    save the result to a variable of your choice. What happens to your
    data?
  \item
    Calculate the mean of the 371st column of
    \texttt{big.exp.draws.1.mat}.
  \item
    Now, find the means of all 1000 columns of
    \texttt{big.exp.draws.1.mat} simultaneously. Plot the histogram of
    column means. Explain why its shape does not match the histogram in
    problem 5b).
  \item
    Take the square of each number in \texttt{big.exp.draws.1}, and find
    the mean of this new vector. Explain this in terms of the mean and
    standard deviation of \texttt{big.exp.draws.1}.
    \textbf{\emph{Hint:}} think carefully about the formula R uses to
    calculate the standard deviation.
  \end{enumerate}
\end{enumerate}

\end{document}
